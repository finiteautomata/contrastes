In this multidisciplinary work, we developed and compared three novel metrics to detect regionalisms on Twitter based on Information Theory. One of these metric was based on the occurrences of a word (\wordmetric), another based on the number of users of a word (\usermetric), and the third combined the other two (\mixedmetric). A team of lexicographers annotated the first thousand words as ranked by each of these metrics to assess the presence of regionalisms, and \usermetric{} yielded the best results, suggesting that the number of users of a word is more informative than its usage frequency.

This method aided lexicographers in their task, letting them
%add a number of words into the 2019 version of 
propose the addition of a number of words into 
%%anonimizacion, cambiar en la version final
the \emph{Diccionario del Habla de los Argentinos}. In the case of this particular dictionary, work relies on a collaborative effort that is based on the intuition of academics and lexicographers that identify regionalisms used mainly (seldom exclusively) within Argentina's borders by carefully parsing over a diversity of sources. This method, albeit effective, can hardly aspire to be comprehensive. Therefore, the promise shown by the present approach does not limit itself to avoiding most of this manual work, which, in and of itself, would already be a sizeable contribution. Since a considerable portion of the lexical repertoire of a community does not make its way across to published materials (which make most of the 300 millions words included to date in, for example, CORPES XXI \cite{espanolabanco}), the possibility of creating lists of words that are likely to be regional, based on actual written utterances by users, opens a way of shedding light onto entire pockets of lexical items that would remain otherwise chronically underrepresented in dictionaries. Even when a regional word is published, and then included in corpora, the task of appropriately isolating it remains largely unchanged, given that the word has to previously be identified in order to then take advantage of the statistical information available. Should the opportunity present itself, the potential for using Information Theory in this way with corpora like CORPES XXI is very significant.

Finally, a challenge triggered by this work is the detection of regions with different dialectal uses as done in previous works \cite{gonccalves2014crowdsourcing}, but using features obtained in a semisupervised fashion with our metrics. This would allow to assess the validity of the dialectal regions of Argentina proposed by Vidal de Battini in 1964 \cite{vidal1964espanol}. Also, spatial and temporal information could be explored to further enhance the precision of this method. 

