Lexicography is the art of writing (designing, compiling, editing) dictionaries: that is, the description of the vocabulary used by members of a speech community \cite{atkins2008oxford}. In the last 30 years, tools coming from Computational Linguistics have helped with this kind of work, mainly in the form of corpora of selected texts. Statistical analysis of corpora results in evidence to support the addition or removal of a word from a dictionary, its marking as dated or unused, as regional, etc., depending on different criteria.

In the process of compiling dictionaries, differences emerge between dialects, where frequently certain words or meanings do not span across all speakers. Since languages are ideal constructs based on the observation of dialects, it is of paramount importance to establish which words are most likely to be shared by an entire linguistic community and which are only used by a smaller group. In this last case, the description profits greatly from information as precise as possible, about geographical extension (region, province, district, city, even neighborhood), about registry (colloquial, neutral, formal), about frequency (actual, past or a combination of both depending on chronological span of the corpus), or any other variable.

 Words that are used exclusively or mainly in a particular subregion of the territory occupied by a linguistic community, or that are used there with a different meaning, are called \emph{regionalisms}. For example, the words ``che''\footnote{Interjection used to get the interlocutor's attention.} and ``metegol''\footnote{Mechanic game that emulates football (\textit{futbolín}) \cite{academia2008diccionario}.} are used more frequently in Argentina than in Spain. Such words are commonly detected through surveys \cite{almeida1995variacion, labov2005atlas} or transcriptions, using methods that depend more or less on the intuition and expertise of linguists. The results of this methodology are of great value to lexicographers, who need evidence to support either the addition of a word into a regional dictionary or the indication of where it is used. Information gathered with these traditional methods has been used as lexical variables to computationally calculate similarities in dialects \cite{kessler1995computational, nerbonne1996phonetic}. 

The irruption of Social Media and its microblogging services has produced an unprecedented wealth of content, with a clear tendency towards informal or colloquial text generated by users. This opens many opportunities to linguists due to the possibility of accessing geotagged contents, which provide valuable information about the origin of users. Social media texts have been used to study dialects and establish ``continuous'' isoglosses \cite{gonccalves2014crowdsourcing, huang2016understanding}. These works typically use features gathered from sources such as web searches \cite{grieve2013site} and even manually-collected regionalisms \cite{ueda2003varilex}. 
Only very recently, methods to automatically extract regionalisms from Twitter have been explored using \emph{tf-idf} and Space Correlation metrics \cite{jimenez2018automatic}.

A valuable framework to navigate this ocean of data is Information Theory. Tools from this field have been used to tell whether a hashtag is promoted by spammers by analyzing its dispersion in time and users \cite{Cui:2012:DBE:2396761.2398519, ghosh2011entropy}, and also to discover valuable features from users messages on Twitter for sentiment analysis and opinion mining \cite{pak2010twitter}.

In the present work, we introduce a quantitative method based on Information Theory to detect regionalisms in Social Media Texts, particularly on Twitter. We present three novel metrics to identify words having different usage in different regions of Argentina. The first metric is based on just words counts; the second, on the numbers of users of words; and the third combines the other two. 81M tweets from  56K users from Argentina were analyzed province-wise, and rankings of the vocabulary were created for each metric. Finally, a qualitative analysis was performed to detect regionalisms and to evaluate each metric.