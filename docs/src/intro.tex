Lexicography is the art of writing (designing, compiling, editing) dictionaries: that is, the description of the vocabulary used by members of a speech community \cite{atkins2008oxford}. This work has been aided in the last 30 years by tools coming from Computational Linguistics, mainly in the form of corpora of selected texts. Statistical analysis of corpora results in evidence to support the addition or removal of a word from a dictionary, its marking as dated or unused, as regional, etc., depending on different criteria.

In the process of compiling dictionaries, differences emerge between dialects, where frequently certain words or meanings do not span across all speakers. \todo{Agregar algo de Español acá, no? es donde los dialectos tienen más importancia} Since languages are an ideal construct based on the observation of dialects, it is of paramount importance to establish which words are most likely to be shared by an entire linguistic community and which are only used by a smaller group. In this last case, the description profits greatly from information as precise as possible, about geographical extension (region, province, district, city, even neighborhood), about registry (colloquial, neutral, formal), about frequency (actual, past or a combination of both depending on chronological span of the corpus), or any other variable.

Words that are used exclusively or mainly in a particular subregion of the territory occupied by a linguistic community, or that are used there with a different meaning, are called \emph{regionalisms}\todo{buscar referencia para esto y chequear}. For example. the word ``che''\footnote{interjection used to get the interlocutor's attention}, or ``metegol''\footnote{mechanic game that emulates footbal (futbolín) \cite{academia2008diccionario}.} are terms used more frequently in Argentina than in Spain. These words are commonly detected through surveys \cite{almeida1995variacion, labov2005atlas}, or transcriptions, using methods depending more or less on the intuition and expertise of linguists. The result of this work is of great value for lexicographers, as they need evidence to support either the addition of a word into a regional dictionary or the indication of where the word is used. Information gathered with these traditional methods has been used to computationally calculate similarities in dialects \cite{kessler1995computational, nerbonne1996phonetic}. 

The irruption of Social Media and its microblogging services produced an unprecedented wealth of content, mainly informal text generated by users. In particular, Twitter gives a great opportunity to linguists due to the possibility of accessing to geotagged tweets, and gathering information about the procedence of users. This has been used to study dialects of languages\cite{gonccalves2014crowdsourcing,huang2016understanding} and establish ``continuous'' isoglosses of them.

An valuable framework to navigate in this ocean of data is Information Theory. Tools from this field have been used to tell whether a hashtag is promoted by spammers \cite{Cui:2012:DBE:2396761.2398519, ghosh2011entropy} by analyzing its dispersion in time and users; also, to detect valuable features in Sentiment Analysis for statuses in this microblogging platform\cite{pak2010twitter}.

In the present work, we introduce a quantitative method based on Information Theory to find these regionalisms in Social Media Texts, particularly on Twitter. This method aided lexicographers in their task, avoiding most of the manual work described, and let them add a number of words into the 2018 version of \emph{Diccionario del Habla de los Argentinos}.\todo{expandir un poco}