
\label{palabras_candidatas}





Based on the list of words identified as having a contrastive use in a region, we performed a linguistic validation of the first thousand words given by each metric. It consisted of a detailed study, word by word, to determine if the word in question is part of the lexical repertoire of a community of speakers. 

This excluded, as is traditional in lexicography, proper and local place names. These words occur mostly in their respective regions, having high entropy and also a high number of occurrences, hence resulting in high values for our metrics. To facilitate the detection of regionalisms, we automatically highlighted words suspected to be toponyms so that the team of lexicographers have a first warning about it.

A team of lexicographers analyzed the first thousand words for each metric. Along these lists, they were provided with tables having figures for each word and province: users, occurrences and normalized frequency (ocurrences per million words). Table \ref{tab:metric_results} shows the results of this labeling. Using information of the number of users of a word seems to be capturing the highest number of words marked as regionalisms. 

\todo{Agregamos algo sobre que "exploramos más palabras" que las primeras 1000? o lo dejamos ahí?}
From these words marked as regionalisms, lexicographers performed a characterization of the results according to the linguistic phenomenon they represent. Table \ref{tab:characterisation} displays lexicographic groups among the regionalisms found in the analyzed words with examples. Table \ref{tab:guaranitic} displays occurrences of three outstanding examples: words coming from guaranitic region.

\todo{Que agregue algo más Santiago sobre esto}. 



\todo{Chequear estos números!!!!}
\begin{table}[hb]
    \centering
    \begin{tabular}{l c c}
    Word & Guaranitic Region & Litoral Region \\
    \hline
    angá & 32.80 & 0.33 \\
    angaú & 8.42 & 0.03 \\
    mitaí & 15.06 & 0.04 \\
    \hline
    \end{tabular}
    \caption{Guaranitic words and its occurrences per million in two different dialectal regions}
    \label{tab:guaranitic}
\end{table}




\begin{table*}[ht!]
\centering

\begin{tabular}{p{0.15\textwidth} p{0.1\textwidth} p{0.65\textwidth}}

Group               & Province & Example  \\ % &           STD \\

\hline
Colloquialisms      & Córdoba & \blockquote{Perdon pero tenes que ser muy \textbf{culiado} para ir a mc y pedirte una ensalada}   \\
& Mendoza & \blockquote{Q \textbf{chombi} hacer un chiste y q la otra persona no se ría o no lo entienda} \\
& Formosa & \blockquote{Tenía la re expectativa para este sábado y al final \textbf{trancó} todo} \\


\hline
Colloquialisms with a verbal or noun base& Buenos Aires &   \blockquote{Me vine a acostar y ya me dicen que parezco de 80 años ME CHUPA UN HUEVO LO QUE PIENSEN, DEJENME \textbf{ABUELEAR} } \\

& Neuquén & \blockquote{Me calma mucho \textbf{mimosear} a mi perro } \\

& Tierra del Fuego & \blockquote{Estaría bueno que ari venga aunque sea a saludarme y que no se quede todo el tiempo \textbf{pollereando}.} \\


\hline \\

Indigenisms        &  Formosa & \blockquote{Te regalo ser \textbf{mitaí} y ir a jurar la bandera con el guardapolvo caliente ese y la corbata que te ahorca todo} \\
& Corrientes & \textbf{Angá} mi negrito, esta triste \\
& Tucumán & Gracias tormenta \textbf{ura} por sonar como una pochoclera de chasquibums a las 3 de la mañana en mi ventana \\
\hline

Words alluding regional reality & San Juan & \blockquote{Quiero a alguien que me diga vamos a comer \textbf{piadinas}, un pancho, un chori, una hamburguesa lo que sea y soy feliz} \\

& Misiones & \blockquote{\textbf{Tareferos} que reclamaban asistencia interzafra en Posadas estarían preparando una protesta para hoy en la Fiesta del Inmigrante en Oberá.} \\
& Jujuy & \blockquote{Me encantan los bohemios anti sistema que usan vans. Es como que seas ecologista y uses un cuaderno hecho con media \textbf{yunga}.} \\


\hline

``Leism'' & Misiones &  \blockquote{No te olvides de \textbf{saludarle} a tu suegro hoy} \\

& Misiones &   \blockquote{Vine a \textbf{visitarle} a mis primas y estan re colgadas, para eso me quedaba en mi casa no maaa } \\

& Formosa & \blockquote{A \textbf{esperarle} a nahuel, que traiga los teresss } \\

\hline
Fusions and acronyms & Buenos Aires & \blockquote{Los sueños de la siesta me dejan \textbf{patra} }\\

%   \blockquote[Córdoba]{Si mañana me dice q no, voy sola, necesito ver esa pelicula en el cine siosi}

\hline

Words with different meaning in a region & Mendoza &  \blockquote{Mañana que alguien \textbf{atine} con parque y porrones} \\

& San Juan & \blockquote{\textbf{Mansas} ganas de sentarme a tomar un te con semitas} \\

& Tierra del Fuego &  \blockquote{\textbf{Habilítenme} una nueva espaldaa} \\

& San Juan &  \blockquote{sigo \textbf{asada} por cosas que han pasado hace como dos dias, que falla (Mendoza) / Que \textbf{asada} estoy, tengo la cabeza echa un lío} \\


\hline

Intejerctions & Formosa &   \blockquote{\textbf{Aijué}, encima me decís vieja, re que no pinta esto facundo jaja ya te dije como es la onda, fin } \\

& Formosa &  \blockquote{\textbf{Ains}, una mujer hablando de fútbol.} \\

& Corrientes & \blockquote{Al fin una buena: hora libreeee! \textbf{Yirr} } \\
 

\hline 
\end{tabular}
\caption{Characterisation of the regionalisms found in the analysis.  }

\label{tab:characterisation}
\end{table*}


