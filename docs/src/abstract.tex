The task of detecting regional colloquialisms —expressions or words used in certain regions in informal language— has classically relied on the use of questionnaires and surveys, and has also heavily depended on the expertise and intuition of the lexicographer.

The irruption of Social Media and its microblogging services produced an unprecedented wealth of content, mainly informal text generated by users. This gives an incredible opportunity for linguists to perform studies about the use and variation of languages.

In this work, we present three metrics based on Information Theory to detect regional colloquialisms on Twitter. The metrics proposed take into account both the number of occurrences of the word in a certain region and the number of users of the word. This tool helped lexicographers to discover a number of unregistered words of Argentinian Spanish, and also different meanings assigned to known expressions.
