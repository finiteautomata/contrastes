
Although there are no other projects that provide a comparison term to assess the degree of success involved in this relationship, there is no doubt that, at least in the detection of local colloquialisms \todo{cambiar a "regionalism"}currently in use, the tool poses a real point of inflection for contrastive lexicography. This area of the lexicon is just the most elusive, since its impact on any printed medium arrives noticeably later and, even more important, in most cases it never reaches. Several words that are already included in the Dictionary of the Speech of Argentines \cite {academia2008diccionario} were included as relevant, given that this fact is an additional confirmation of the relevance of the location that assigned the metric.


It is worth mentioning that words coming from \emph{guaraní} —language spoken in Northern Argentina, Paraguay, Bolivia and Southwest of Brazil— coincide with the region delimited by \cite{vidal1964espanol}. An example of this phenomenon are the words \textit{angá}, \textit{angaú} y \textit{mitaí}.

- Agregar que esto aplica no sólo a encontrar regionalismos sino temas/términos que sean propios de una región geográfica particular (marketing, ver penetración de una marca como en una de las citas).